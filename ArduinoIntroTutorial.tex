


% Header, overrides base

    % Make sure that the sphinx doc style knows who it inherits from.
    \def\sphinxdocclass{article}

    % Declare the document class
    \documentclass[letterpaper,10pt,english]{/usr/local/lib/python2.7/dist-packages/sphinx/texinputs/sphinxhowto}

    % Imports
    \usepackage[utf8]{inputenc}
    \DeclareUnicodeCharacter{00A0}{\\nobreakspace}
    \usepackage[T1]{fontenc}
    \usepackage{babel}
    \usepackage{times}
    \usepackage{import}
    \usepackage[Bjarne]{/usr/local/lib/python2.7/dist-packages/sphinx/texinputs/fncychap}
    \usepackage{longtable}
    \usepackage{/usr/local/lib/python2.7/dist-packages/sphinx/texinputs/sphinx}
    \usepackage{multirow}

    \usepackage{amsmath}
    \usepackage{amssymb}
    \usepackage{ucs}
    \usepackage{enumerate}

    % Used to make the Input/Output rules follow around the contents.
    \usepackage{needspace}

    % Pygments requirements
    \usepackage{fancyvrb}
    \usepackage{color}
    % ansi colors additions
    \definecolor{darkgreen}{rgb}{.12,.54,.11}
    \definecolor{lightgray}{gray}{.95}
    \definecolor{brown}{rgb}{0.54,0.27,0.07}
    \definecolor{purple}{rgb}{0.5,0.0,0.5}
    \definecolor{darkgray}{gray}{0.25}
    \definecolor{lightred}{rgb}{1.0,0.39,0.28}
    \definecolor{lightgreen}{rgb}{0.48,0.99,0.0}
    \definecolor{lightblue}{rgb}{0.53,0.81,0.92}
    \definecolor{lightpurple}{rgb}{0.87,0.63,0.87}
    \definecolor{lightcyan}{rgb}{0.5,1.0,0.83}

    % Needed to box output/input
    \usepackage{tikz}
        \usetikzlibrary{calc,arrows,shadows}
    \usepackage[framemethod=tikz]{mdframed}

    \usepackage{alltt}

    % Used to load and display graphics
    \usepackage{graphicx}
    \graphicspath{ {figs/} }
    \usepackage[Export]{adjustbox} % To resize

    % used so that images for notebooks which have spaces in the name can still be included
    \usepackage{grffile}


    % For formatting output while also word wrapping.
    \usepackage{listings}
    \lstset{breaklines=true}
    \lstset{basicstyle=\small\ttfamily}
    \def\smaller{\fontsize{9.5pt}{9.5pt}\selectfont}

    %Pygments definitions
    
\makeatletter
\def\PY@reset{\let\PY@it=\relax \let\PY@bf=\relax%
    \let\PY@ul=\relax \let\PY@tc=\relax%
    \let\PY@bc=\relax \let\PY@ff=\relax}
\def\PY@tok#1{\csname PY@tok@#1\endcsname}
\def\PY@toks#1+{\ifx\relax#1\empty\else%
    \PY@tok{#1}\expandafter\PY@toks\fi}
\def\PY@do#1{\PY@bc{\PY@tc{\PY@ul{%
    \PY@it{\PY@bf{\PY@ff{#1}}}}}}}
\def\PY#1#2{\PY@reset\PY@toks#1+\relax+\PY@do{#2}}

\expandafter\def\csname PY@tok@gd\endcsname{\def\PY@tc##1{\textcolor[rgb]{0.63,0.00,0.00}{##1}}}
\expandafter\def\csname PY@tok@gu\endcsname{\let\PY@bf=\textbf\def\PY@tc##1{\textcolor[rgb]{0.50,0.00,0.50}{##1}}}
\expandafter\def\csname PY@tok@gt\endcsname{\def\PY@tc##1{\textcolor[rgb]{0.00,0.27,0.87}{##1}}}
\expandafter\def\csname PY@tok@gs\endcsname{\let\PY@bf=\textbf}
\expandafter\def\csname PY@tok@gr\endcsname{\def\PY@tc##1{\textcolor[rgb]{1.00,0.00,0.00}{##1}}}
\expandafter\def\csname PY@tok@cm\endcsname{\let\PY@it=\textit\def\PY@tc##1{\textcolor[rgb]{0.25,0.50,0.50}{##1}}}
\expandafter\def\csname PY@tok@vg\endcsname{\def\PY@tc##1{\textcolor[rgb]{0.10,0.09,0.49}{##1}}}
\expandafter\def\csname PY@tok@m\endcsname{\def\PY@tc##1{\textcolor[rgb]{0.40,0.40,0.40}{##1}}}
\expandafter\def\csname PY@tok@mh\endcsname{\def\PY@tc##1{\textcolor[rgb]{0.40,0.40,0.40}{##1}}}
\expandafter\def\csname PY@tok@go\endcsname{\def\PY@tc##1{\textcolor[rgb]{0.53,0.53,0.53}{##1}}}
\expandafter\def\csname PY@tok@ge\endcsname{\let\PY@it=\textit}
\expandafter\def\csname PY@tok@vc\endcsname{\def\PY@tc##1{\textcolor[rgb]{0.10,0.09,0.49}{##1}}}
\expandafter\def\csname PY@tok@il\endcsname{\def\PY@tc##1{\textcolor[rgb]{0.40,0.40,0.40}{##1}}}
\expandafter\def\csname PY@tok@cs\endcsname{\let\PY@it=\textit\def\PY@tc##1{\textcolor[rgb]{0.25,0.50,0.50}{##1}}}
\expandafter\def\csname PY@tok@cp\endcsname{\def\PY@tc##1{\textcolor[rgb]{0.74,0.48,0.00}{##1}}}
\expandafter\def\csname PY@tok@gi\endcsname{\def\PY@tc##1{\textcolor[rgb]{0.00,0.63,0.00}{##1}}}
\expandafter\def\csname PY@tok@gh\endcsname{\let\PY@bf=\textbf\def\PY@tc##1{\textcolor[rgb]{0.00,0.00,0.50}{##1}}}
\expandafter\def\csname PY@tok@ni\endcsname{\let\PY@bf=\textbf\def\PY@tc##1{\textcolor[rgb]{0.60,0.60,0.60}{##1}}}
\expandafter\def\csname PY@tok@nl\endcsname{\def\PY@tc##1{\textcolor[rgb]{0.63,0.63,0.00}{##1}}}
\expandafter\def\csname PY@tok@nn\endcsname{\let\PY@bf=\textbf\def\PY@tc##1{\textcolor[rgb]{0.00,0.00,1.00}{##1}}}
\expandafter\def\csname PY@tok@no\endcsname{\def\PY@tc##1{\textcolor[rgb]{0.53,0.00,0.00}{##1}}}
\expandafter\def\csname PY@tok@na\endcsname{\def\PY@tc##1{\textcolor[rgb]{0.49,0.56,0.16}{##1}}}
\expandafter\def\csname PY@tok@nb\endcsname{\def\PY@tc##1{\textcolor[rgb]{0.00,0.50,0.00}{##1}}}
\expandafter\def\csname PY@tok@nc\endcsname{\let\PY@bf=\textbf\def\PY@tc##1{\textcolor[rgb]{0.00,0.00,1.00}{##1}}}
\expandafter\def\csname PY@tok@nd\endcsname{\def\PY@tc##1{\textcolor[rgb]{0.67,0.13,1.00}{##1}}}
\expandafter\def\csname PY@tok@ne\endcsname{\let\PY@bf=\textbf\def\PY@tc##1{\textcolor[rgb]{0.82,0.25,0.23}{##1}}}
\expandafter\def\csname PY@tok@nf\endcsname{\def\PY@tc##1{\textcolor[rgb]{0.00,0.00,1.00}{##1}}}
\expandafter\def\csname PY@tok@si\endcsname{\let\PY@bf=\textbf\def\PY@tc##1{\textcolor[rgb]{0.73,0.40,0.53}{##1}}}
\expandafter\def\csname PY@tok@s2\endcsname{\def\PY@tc##1{\textcolor[rgb]{0.73,0.13,0.13}{##1}}}
\expandafter\def\csname PY@tok@vi\endcsname{\def\PY@tc##1{\textcolor[rgb]{0.10,0.09,0.49}{##1}}}
\expandafter\def\csname PY@tok@nt\endcsname{\let\PY@bf=\textbf\def\PY@tc##1{\textcolor[rgb]{0.00,0.50,0.00}{##1}}}
\expandafter\def\csname PY@tok@nv\endcsname{\def\PY@tc##1{\textcolor[rgb]{0.10,0.09,0.49}{##1}}}
\expandafter\def\csname PY@tok@s1\endcsname{\def\PY@tc##1{\textcolor[rgb]{0.73,0.13,0.13}{##1}}}
\expandafter\def\csname PY@tok@sh\endcsname{\def\PY@tc##1{\textcolor[rgb]{0.73,0.13,0.13}{##1}}}
\expandafter\def\csname PY@tok@sc\endcsname{\def\PY@tc##1{\textcolor[rgb]{0.73,0.13,0.13}{##1}}}
\expandafter\def\csname PY@tok@sx\endcsname{\def\PY@tc##1{\textcolor[rgb]{0.00,0.50,0.00}{##1}}}
\expandafter\def\csname PY@tok@bp\endcsname{\def\PY@tc##1{\textcolor[rgb]{0.00,0.50,0.00}{##1}}}
\expandafter\def\csname PY@tok@c1\endcsname{\let\PY@it=\textit\def\PY@tc##1{\textcolor[rgb]{0.25,0.50,0.50}{##1}}}
\expandafter\def\csname PY@tok@kc\endcsname{\let\PY@bf=\textbf\def\PY@tc##1{\textcolor[rgb]{0.00,0.50,0.00}{##1}}}
\expandafter\def\csname PY@tok@c\endcsname{\let\PY@it=\textit\def\PY@tc##1{\textcolor[rgb]{0.25,0.50,0.50}{##1}}}
\expandafter\def\csname PY@tok@mf\endcsname{\def\PY@tc##1{\textcolor[rgb]{0.40,0.40,0.40}{##1}}}
\expandafter\def\csname PY@tok@err\endcsname{\def\PY@bc##1{\setlength{\fboxsep}{0pt}\fcolorbox[rgb]{1.00,0.00,0.00}{1,1,1}{\strut ##1}}}
\expandafter\def\csname PY@tok@kd\endcsname{\let\PY@bf=\textbf\def\PY@tc##1{\textcolor[rgb]{0.00,0.50,0.00}{##1}}}
\expandafter\def\csname PY@tok@ss\endcsname{\def\PY@tc##1{\textcolor[rgb]{0.10,0.09,0.49}{##1}}}
\expandafter\def\csname PY@tok@sr\endcsname{\def\PY@tc##1{\textcolor[rgb]{0.73,0.40,0.53}{##1}}}
\expandafter\def\csname PY@tok@mo\endcsname{\def\PY@tc##1{\textcolor[rgb]{0.40,0.40,0.40}{##1}}}
\expandafter\def\csname PY@tok@kn\endcsname{\let\PY@bf=\textbf\def\PY@tc##1{\textcolor[rgb]{0.00,0.50,0.00}{##1}}}
\expandafter\def\csname PY@tok@mi\endcsname{\def\PY@tc##1{\textcolor[rgb]{0.40,0.40,0.40}{##1}}}
\expandafter\def\csname PY@tok@gp\endcsname{\let\PY@bf=\textbf\def\PY@tc##1{\textcolor[rgb]{0.00,0.00,0.50}{##1}}}
\expandafter\def\csname PY@tok@o\endcsname{\def\PY@tc##1{\textcolor[rgb]{0.40,0.40,0.40}{##1}}}
\expandafter\def\csname PY@tok@kr\endcsname{\let\PY@bf=\textbf\def\PY@tc##1{\textcolor[rgb]{0.00,0.50,0.00}{##1}}}
\expandafter\def\csname PY@tok@s\endcsname{\def\PY@tc##1{\textcolor[rgb]{0.73,0.13,0.13}{##1}}}
\expandafter\def\csname PY@tok@kp\endcsname{\def\PY@tc##1{\textcolor[rgb]{0.00,0.50,0.00}{##1}}}
\expandafter\def\csname PY@tok@w\endcsname{\def\PY@tc##1{\textcolor[rgb]{0.73,0.73,0.73}{##1}}}
\expandafter\def\csname PY@tok@kt\endcsname{\def\PY@tc##1{\textcolor[rgb]{0.69,0.00,0.25}{##1}}}
\expandafter\def\csname PY@tok@ow\endcsname{\let\PY@bf=\textbf\def\PY@tc##1{\textcolor[rgb]{0.67,0.13,1.00}{##1}}}
\expandafter\def\csname PY@tok@sb\endcsname{\def\PY@tc##1{\textcolor[rgb]{0.73,0.13,0.13}{##1}}}
\expandafter\def\csname PY@tok@k\endcsname{\let\PY@bf=\textbf\def\PY@tc##1{\textcolor[rgb]{0.00,0.50,0.00}{##1}}}
\expandafter\def\csname PY@tok@se\endcsname{\let\PY@bf=\textbf\def\PY@tc##1{\textcolor[rgb]{0.73,0.40,0.13}{##1}}}
\expandafter\def\csname PY@tok@sd\endcsname{\let\PY@it=\textit\def\PY@tc##1{\textcolor[rgb]{0.73,0.13,0.13}{##1}}}

\def\PYZbs{\char`\\}
\def\PYZus{\char`\_}
\def\PYZob{\char`\{}
\def\PYZcb{\char`\}}
\def\PYZca{\char`\^}
\def\PYZam{\char`\&}
\def\PYZlt{\char`\<}
\def\PYZgt{\char`\>}
\def\PYZsh{\char`\#}
\def\PYZpc{\char`\%}
\def\PYZdl{\char`\$}
\def\PYZhy{\char`\-}
\def\PYZsq{\char`\'}
\def\PYZdq{\char`\"}
\def\PYZti{\char`\~}
% for compatibility with earlier versions
\def\PYZat{@}
\def\PYZlb{[}
\def\PYZrb{]}
\makeatother


    %Set pygments styles if needed...
    
        \definecolor{nbframe-border}{rgb}{0.867,0.867,0.867}
        \definecolor{nbframe-bg}{rgb}{0.969,0.969,0.969}
        \definecolor{nbframe-in-prompt}{rgb}{0.0,0.0,0.502}
        \definecolor{nbframe-out-prompt}{rgb}{0.545,0.0,0.0}

        \newenvironment{ColorVerbatim}
        {\begin{mdframed}[%
            roundcorner=1.0pt, %
            backgroundcolor=nbframe-bg, %
            userdefinedwidth=1\linewidth, %
            leftmargin=0.1\linewidth, %
            innerleftmargin=0pt, %
            innerrightmargin=0pt, %
            linecolor=nbframe-border, %
            linewidth=1pt, %
            usetwoside=false, %
            everyline=true, %
            innerlinewidth=3pt, %
            innerlinecolor=nbframe-bg, %
            middlelinewidth=1pt, %
            middlelinecolor=nbframe-bg, %
            outerlinewidth=0.5pt, %
            outerlinecolor=nbframe-border, %
            needspace=0pt
        ]}
        {\end{mdframed}}
        
        \newenvironment{InvisibleVerbatim}
        {\begin{mdframed}[leftmargin=0.1\linewidth,innerleftmargin=3pt,innerrightmargin=3pt, userdefinedwidth=1\linewidth, linewidth=0pt, linecolor=white, usetwoside=false]}
        {\end{mdframed}}

        \renewenvironment{Verbatim}[1][\unskip]
        {\begin{alltt}\smaller}
        {\end{alltt}}
    

    % Help prevent overflowing lines due to urls and other hard-to-break 
    % entities.  This doesn't catch everything...
    \sloppy

    % Document level variables
    \title{ArduinoIntroTutorial}
    \date{November 24, 2013}
    \release{}
    \author{Michele Pratusevich}
    \renewcommand{\releasename}{}

    % TODO: Add option for the user to specify a logo for his/her export.
    \newcommand{\sphinxlogo}{}

    % Make the index page of the document.
    \makeindex

    % Import sphinx document type specifics.
     


% Body

    % Start of the document
    \begin{document}

        
            \maketitle
        

        


        
        \section{MEET YL2 Individual Project: The Arduino Tutorial }

\subsection{The plan for you, reader of this tutorial}

Your instructions are to work through this tutorial. If you don't
understand something, ask each other for help. Only come ask me if
something is unclear and no one can figure it out. Of it you think you
are doing something that is dangerous to you and / or others. The
tutorial takes you through roughly these steps:

\begin{enumerate}[1.]
\item
  \href{\#hardware}{Read about hardware, Arduinos}
\item
  \href{\#simpleprogram}{Play with the Arduino-program-language with
  some simple programs}
\item
  \href{\#serial}{Play with serial connections to the computer}
\item
  \href{\#breadboards}{Learn how breadboards work}
\item
  \href{\#button}{Learn how to attach a button to an Arduino}
\item
  \href{\#indivproj}{Plan your individual project}
\item
  Work work work!
\end{enumerate}\subsection{What is hardware? Why do we care? (What you CAN DO with
hardware) \#\# (top)}

\begin{itemize}
\item
  Interact with the physical world - react, make something move, etc.
  \begin{itemize}
  \item
    User presses buttons, turns dials
  \item
    Sensors get information from environment
  \item
    Control motors
  \end{itemize}
\item
  Run code on tiny computers that can stand alone (without a full
  computer)
\end{itemize}
\subsection{What you WILL DO with hardware}

\begin{itemize}
\item
  We won't control anything - for now we will focus on getting input,
  either from the user or from the physical world.
\item
  To start with, you will work on getting input from the user through a
  physical external button press. If there is time, you will use some
  other sensor to get information from the user or the environment in a
  different way.
\end{itemize}\subsection{A bit about hardware in general}

\emph{Hardware} is a broad term used to describe physical components
outside of a computer. This includes electronics, circuits, and chips
(the things that make the inside of a computer). Some facts, terms,
information:

\begin{itemize}
\item
  Hardware is serious business. It is sensitive physical pieces that
  carry electricity to perform a certain task. Connecting wires the
  wrong way, putting physical stress on the parts, or making the parts
  dirty, wet, sticky, etc. is a danger to your physical safety. Please
  respect the hardware, and make sure to take good care of it, both for
  it's sake and for your sake.
\item
  \textbf{microcontroller} - a small computer, usually without a
  display. Has limited code that it understands, and is often coded for
  a specific purpose in the physical world away from a computer. It,
  like any other computer, needs to get power from somewhere.
\item
  \textbf{microprocessor} - another word for a small computer, but one
  that is more powerful than a microcontroller
\item
  Since you can't directly type code onto a microcontroller, you type
  your code in a computer, then compile and upload it (``burn'' it) onto
  the microcontroller. Usually you do this through a USB cable that
  talks with the microcontroller using a language called \textbf{serial
  communication}. If you tell the microcontroller to talk to the
  computer, it will do so through the \textbf{serial console}.
\item
  Another name for the USB port is a \textbf{COM port}.
\end{itemize}
Some electronics concepts you will need to learn about hardware:

\begin{itemize}
\item
  \textbf{Components} are pieces of electrical hardware, like resistors,
  buttons, and wires
\item
  \textbf{Digital} vs. \textbf{analog} - When something is digital, it
  has two values it can take, ON (1) or OFF (0). There is no such thing
  as in-between. When something is analog, it can take any value between
  ON (1) and OFF (0). Digital inputs and outputs are used for detecting
  things like button presses and for counting numbers, while analog
  inputs are used for detecting things like dial or slider position
\item
  \textbf{Sensors} are electrical components that measure something.
  They can be temperature sensors, distance sensors, light sensors,
  sound sensors, etc.
\item
  In the world of both digital and analog electronics in
  microprocessors, the maximum (ON) voltage is usually 5 volts, and this
  is referred to as \textbf{power}. The opposite, or OFF voltage, is 0
  volts, and is referred to as \textbf{ground}, often abbreviated
  \textbf{GND}.
\item
  A \textbf{breadboard} is a grid used putting together electrical
  components. Like this:
\end{itemize}\subsection{Arduino: the easiest way to get started }

The best way to get started in getting comfortable with these tools is
by using a microcontroller platform that is easy to use and has a lot of
tutorials. This is the \href{http://arduino.cc}{Arduino} platform that
we are going to use.

You can see a few labeled parts on the board. Try to find them on your
own boards.

The cable that connects the Arduino to the computer is a USB cable that
connects to a USB port (it is on the back of the laptop) and to the USB
connector on the Arduino. This USB cable is also the cable that gives
the Arduino electricity. Go ahead and do that now. You should see the
red ``power'' light be on in the Arduino.\subsubsection{What is on the Arduino and what is on the computer}

The workflow is like this: you write code on the computer in an editor,
compile it on the computer, and send it to the Arduino (using the serial
protocol) through the cable. As soon as this process is complete, the
code executes immediately on the Arduino. The Arduino can then talk to
the computer (if you tell it to) by printing things to the computer
through the \textbf{serial console}. The way to ``restart'' a program
running on the Arduino is to press the \textbf{reset button}.\subsection{How we work with Arduino and the Arduino IDE \#\# (top)}

\begin{itemize}
\item
  Coding our Arduino programs on Ubuntu using the Arduino IDE
  (\textbf{IDE} stands for integrated development environment, which
  basically means a program that gives us a place to write our code)
\item
  Arduino uses a special language; it looks like C++, but it reads like
  Python (and like English). The best way to learn the language is to
  look at examples and play with the code yourself. So there will not be
  an explanation of how the code works - you will learn by going through
  the example code. After you finish coding your program, the way to
  test it is to upload it to the Arduino and make sure it does what you
  want it to do.
\end{itemize}
\subsubsection{Opening the Arduino IDE}

It is already installed on the Ubuntu machines. To open it, open the
program explorer and start typing ``Arduino'':

It will open a program like this:

The white space is where you will type your code, and the black is where
the error messages will be shown.\subsubsection{Opening Arduino sample code from the IDE}

One of the awesome things about working with Arduino is the wealth of
resources available for Arduino. These include ``official Arduino sample
code.'' To open the tutorials, go to ``File'' -\textgreater{}
``Examples'', and then going to the category you want. Each tutorial
will tell you which example code to open.\subsection{Tutorial 1: BLINK \#\# (top)}

Open the ``Blink'' example code from the ``1. Basics'' example code
menu. DO NOT attach a resistor and LED to the Arduino - the light on the
Arduino board itself is connected to pin 13, so there is no need to risk
blowing up our LEDs. Look through the code, with
\href{http://arduino.cc/en/Tutorial/Blink}{this tutorial} as a guide.

Upload the code to the Arduino using the upload button (the arrow in the
white circle at the top of the IDE window):

When you are done uploading, it should look like:

If you have any orange text in the black box, something is wrong.

If you have problems with your code not uploading (you'll see errors in
the black error box at the bottom of the IDE), check out some
\href{CommonArduinoProblems.ipynb}{common problems you might have}.

Once you are done reading and understanding this code, do the following
exercises:

\begin{enumerate}[1.]
\item
  Change the light to be on for 2 seconds, off for 2 seconds
\item
  Change the light to be on for 3 seconds, off for 1 second
\end{enumerate}
You don't need to submit anything for these exercises, but you might
want to save a copy of your modified programs in your Github folder.
Answer for yourself the following question: What section of code happens
once? What section of code continues to happen during execution?\subsection{Tutorial 2: Serial communication \#\# (top)}Now, let's play with how to send information from the Arduino to the
computer's screen over serial. Make a new file and paste this code into
it:\begin{verbatim}
/*
 * Hello World!
 *
 * This is the Hello World! for Arduino. 
 * It shows how to send data to the computer
 */


void setup()                    // run once, when the sketch starts
{
  Serial.begin(9600);           // set up Serial library at 9600 bps

  Serial.println("Hello world!");  // prints hello with ending line break 
}

void loop()                       // run over and over again
{
                                  // do nothing!
}
\end{verbatim}Notice that there is nothing in the \texttt{loop()} method, so
everything that happens will happen only once. Upload the code to the
Arduino. Nothing happens. This is because you need to open the Serial
Monitor (the thing that watches what comes to the computer over the
serial port) by going to ``Tools'' -\textgreater{} ``Serial Monitor''.
You should get a window that pops up like this:

To restart the program that has been burned into the Arduino, press and
hold the RESET button on the Arduino (if you don't remember where it is,
go back and look at the \href{\#parts}{parts} diagram). What do you
expect to happen in the Serial Monitor? What actually happens?

Once you've gotten the code to work (and understand the result), do the
following exercises:

\begin{enumerate}[1.]
\item
  Make the Arduino print ``Hello, World!'' over serial once every
  second.
\item
  Make the Arduino print ``Hello, World!'' over serial once every
  second, but only a total of five times.
\end{enumerate}
Save your code from exercise 2 in your Github repository.\subsection{How a breadboard works \#\# (top)}

Remember breadboards?

An interesting thing about breadboards is that the squares in each row
are connected to each other. Horizontal rows are connected:

And the two vertical rows on either side of the board are connected:

For example, if I connect a wire from 5V on the Arduino to cell B2, then
A2, C2, D2, and E2 are all automatically connected to 5V also.

One common way to take advantage of this is to connect one cell in the
right column to GROUND (0 volts), then being able to connect any
components you need to ground by connecting a wire between the component
and the ground column.

For example, if you connect a wire from GND on the Arduino to the side
column on the breadboard, like this:

Then every other square in that column will be connected to GND
automatically.\subsection{Tutorial 3: Buttons \#\# (top)}

Next, open up the DigitalReadSerial, or the Button tutorial. Follow
along with \href{http://arduino.cc/en/Tutorial/DigitalReadSerial}{this
tutorial} to understand the code. Build the circuit, taking care to
connect the appropriate components in the right way. Our buttons look
like this:

And the wires are super-easy to use and look like this:

Test it to make sure the code works.

When you are done, take a picture of your circuit and save it in your
Github repository.\subsection{Your individual project \#\# (top)}

Now that you have seen a few tutorials and hopefully have gotten a
button to work on the Arduino, you need to choose a final project to
work on.

You must choose one of these projects to do as your individual project,
but no two people can do the same project. Projects are written in
approximate order of difficulty, easier projects at the top. If you come
up with a variation of one of these projects, that is fine, just talk to
me to get it approved. Two people can work on a variation of the same
project.

\begin{enumerate}[1.]
\item
  A simple switch - print the number 1 to the screen over and over
  again, until the user presses a button. Then print the number 2 to the
  screen. Every time the user presses the button, increase the number
  displayed to the screen by one.
\item
  Keep-alive Light Button. Pretend that the Arduino button is a light
  that is keepin ga baby chick alive. Keep the Arduino light turned on
  when the button is pressed. Only turn the light off when the button is
  no longer pressed.
\item
  Ask the user to press the button some number of times. After you don't
  hear a button press for 2 seconds, use that number to set how fast the
  LED blinks.
\item
  Make some kind of trivia game: ask the user a question randomly (from
  a fixed list, not from the internet), and the user must press the
  button the correct number of times to answer the question.
\item
  Every two seconds, ask the user for a button press 4 times. Keep track
  of the button presses like this: 1011 (where 1 is a press and 0 is not
  a press). At the end, convert this number from binary to decimal and
  print it to the screen.
\item
  Ask the user to press and hold the button. However long the user
  presses the button, set that to how fast the LED blinks.
\item
  A password detection application - when the user presses the button
  the correct number of times for the correct length of time, the
  password is correctly detected and the computer prints ``success!'' on
  the screen
\item
  A Morse code encoding application - the user can record messages using
  \href{http://en.wikipedia.org/wiki/Morse\_code}{Morse code}, which can
  be printed back to the screen when the user is done
\end{enumerate}Spend some time thinking (and deciding!) on which project you will work
on. To see more information on the project description and requirements,
look
\href{https://docs.google.com/a/meet.mit.edu/document/d/1HzMu8S\_8xAyPVDZaA\_JOZEW5XLl72B5BL6Jtiv3Hze0/edit\#heading=h.kpxwsy50i6vt}{here}.

If you want to see two more tutorials, here are two that come logically
next. Otherwise, start working on your project! But first:

\begin{enumerate}[1.]
\item
  Draw a diagram of any components you need (remember, PLAN, then DO)
\item
  Draw a diagram of any circuit you need for your project
\end{enumerate}\subsection{Tutorial 4: Analog Input (potentiometer)}

A \textbf{potentiometer} is an electrical component that is the
equivalent of a dial or a knob, used to get analog input. Our
potentiometers look like this:

Follow \href{http://arduino.cc/en/Tutorial/AnalogReadSerial}{this}
tutorial to learn how to use them.\subsection{Tutorial 5: Blink without delay}

\href{http://arduino.cc/en/Tutorial/BlinkWithoutDelay}{This} tutorial,
without actually hooking up the LED.\subsection{Setting up Arduino for yourself at home}

Really easy on your own computer. Instructions
\href{http://arduino.cc}{here}. If you want to take one home to work,
talk to me first.\subsection{Resources}

There are a zillion Arduino resources around - there is a lot to look
through and it can be daunting. Here is a list of select resources that
are worth your time, if you need to find anything:

\begin{itemize}
\item
  \href{http://arduino.cc/en/Tutorial/HomePage}{The official Arduino
  tutorials}
\item
  \href{http://hci.rwth-aachen.de/arduino}{Arduino in a Nutshell}
\item
  http://www.ladyada.net/learn/arduino/index.html
\end{itemize}
        

        \renewcommand{\indexname}{Index}
        \printindex

    % End of document
    \end{document}


