


% Header, overrides base

    % Make sure that the sphinx doc style knows who it inherits from.
    \def\sphinxdocclass{article}

    % Declare the document class
    \documentclass[letterpaper,10pt,english]{/usr/local/lib/python2.7/dist-packages/sphinx/texinputs/sphinxhowto}

    % Imports
    \usepackage[utf8]{inputenc}
    \DeclareUnicodeCharacter{00A0}{\\nobreakspace}
    \usepackage[T1]{fontenc}
    \usepackage{babel}
    \usepackage{times}
    \usepackage{import}
    \usepackage[Bjarne]{/usr/local/lib/python2.7/dist-packages/sphinx/texinputs/fncychap}
    \usepackage{longtable}
    \usepackage{/usr/local/lib/python2.7/dist-packages/sphinx/texinputs/sphinx}
    \usepackage{multirow}

    \usepackage{amsmath}
    \usepackage{amssymb}
    \usepackage{ucs}
    \usepackage{enumerate}

    % Used to make the Input/Output rules follow around the contents.
    \usepackage{needspace}

    % Pygments requirements
    \usepackage{fancyvrb}
    \usepackage{color}
    % ansi colors additions
    \definecolor{darkgreen}{rgb}{.12,.54,.11}
    \definecolor{lightgray}{gray}{.95}
    \definecolor{brown}{rgb}{0.54,0.27,0.07}
    \definecolor{purple}{rgb}{0.5,0.0,0.5}
    \definecolor{darkgray}{gray}{0.25}
    \definecolor{lightred}{rgb}{1.0,0.39,0.28}
    \definecolor{lightgreen}{rgb}{0.48,0.99,0.0}
    \definecolor{lightblue}{rgb}{0.53,0.81,0.92}
    \definecolor{lightpurple}{rgb}{0.87,0.63,0.87}
    \definecolor{lightcyan}{rgb}{0.5,1.0,0.83}

    % Needed to box output/input
    \usepackage{tikz}
        \usetikzlibrary{calc,arrows,shadows}
    \usepackage[framemethod=tikz]{mdframed}

    \usepackage{alltt}

    % Used to load and display graphics
    \usepackage{graphicx}
    \graphicspath{ {figs/} }
    \usepackage[Export]{adjustbox} % To resize

    % used so that images for notebooks which have spaces in the name can still be included
    \usepackage{grffile}


    % For formatting output while also word wrapping.
    \usepackage{listings}
    \lstset{breaklines=true}
    \lstset{basicstyle=\small\ttfamily}
    \def\smaller{\fontsize{9.5pt}{9.5pt}\selectfont}

    %Pygments definitions
    
\makeatletter
\def\PY@reset{\let\PY@it=\relax \let\PY@bf=\relax%
    \let\PY@ul=\relax \let\PY@tc=\relax%
    \let\PY@bc=\relax \let\PY@ff=\relax}
\def\PY@tok#1{\csname PY@tok@#1\endcsname}
\def\PY@toks#1+{\ifx\relax#1\empty\else%
    \PY@tok{#1}\expandafter\PY@toks\fi}
\def\PY@do#1{\PY@bc{\PY@tc{\PY@ul{%
    \PY@it{\PY@bf{\PY@ff{#1}}}}}}}
\def\PY#1#2{\PY@reset\PY@toks#1+\relax+\PY@do{#2}}

\expandafter\def\csname PY@tok@gd\endcsname{\def\PY@tc##1{\textcolor[rgb]{0.63,0.00,0.00}{##1}}}
\expandafter\def\csname PY@tok@gu\endcsname{\let\PY@bf=\textbf\def\PY@tc##1{\textcolor[rgb]{0.50,0.00,0.50}{##1}}}
\expandafter\def\csname PY@tok@gt\endcsname{\def\PY@tc##1{\textcolor[rgb]{0.00,0.27,0.87}{##1}}}
\expandafter\def\csname PY@tok@gs\endcsname{\let\PY@bf=\textbf}
\expandafter\def\csname PY@tok@gr\endcsname{\def\PY@tc##1{\textcolor[rgb]{1.00,0.00,0.00}{##1}}}
\expandafter\def\csname PY@tok@cm\endcsname{\let\PY@it=\textit\def\PY@tc##1{\textcolor[rgb]{0.25,0.50,0.50}{##1}}}
\expandafter\def\csname PY@tok@vg\endcsname{\def\PY@tc##1{\textcolor[rgb]{0.10,0.09,0.49}{##1}}}
\expandafter\def\csname PY@tok@m\endcsname{\def\PY@tc##1{\textcolor[rgb]{0.40,0.40,0.40}{##1}}}
\expandafter\def\csname PY@tok@mh\endcsname{\def\PY@tc##1{\textcolor[rgb]{0.40,0.40,0.40}{##1}}}
\expandafter\def\csname PY@tok@go\endcsname{\def\PY@tc##1{\textcolor[rgb]{0.53,0.53,0.53}{##1}}}
\expandafter\def\csname PY@tok@ge\endcsname{\let\PY@it=\textit}
\expandafter\def\csname PY@tok@vc\endcsname{\def\PY@tc##1{\textcolor[rgb]{0.10,0.09,0.49}{##1}}}
\expandafter\def\csname PY@tok@il\endcsname{\def\PY@tc##1{\textcolor[rgb]{0.40,0.40,0.40}{##1}}}
\expandafter\def\csname PY@tok@cs\endcsname{\let\PY@it=\textit\def\PY@tc##1{\textcolor[rgb]{0.25,0.50,0.50}{##1}}}
\expandafter\def\csname PY@tok@cp\endcsname{\def\PY@tc##1{\textcolor[rgb]{0.74,0.48,0.00}{##1}}}
\expandafter\def\csname PY@tok@gi\endcsname{\def\PY@tc##1{\textcolor[rgb]{0.00,0.63,0.00}{##1}}}
\expandafter\def\csname PY@tok@gh\endcsname{\let\PY@bf=\textbf\def\PY@tc##1{\textcolor[rgb]{0.00,0.00,0.50}{##1}}}
\expandafter\def\csname PY@tok@ni\endcsname{\let\PY@bf=\textbf\def\PY@tc##1{\textcolor[rgb]{0.60,0.60,0.60}{##1}}}
\expandafter\def\csname PY@tok@nl\endcsname{\def\PY@tc##1{\textcolor[rgb]{0.63,0.63,0.00}{##1}}}
\expandafter\def\csname PY@tok@nn\endcsname{\let\PY@bf=\textbf\def\PY@tc##1{\textcolor[rgb]{0.00,0.00,1.00}{##1}}}
\expandafter\def\csname PY@tok@no\endcsname{\def\PY@tc##1{\textcolor[rgb]{0.53,0.00,0.00}{##1}}}
\expandafter\def\csname PY@tok@na\endcsname{\def\PY@tc##1{\textcolor[rgb]{0.49,0.56,0.16}{##1}}}
\expandafter\def\csname PY@tok@nb\endcsname{\def\PY@tc##1{\textcolor[rgb]{0.00,0.50,0.00}{##1}}}
\expandafter\def\csname PY@tok@nc\endcsname{\let\PY@bf=\textbf\def\PY@tc##1{\textcolor[rgb]{0.00,0.00,1.00}{##1}}}
\expandafter\def\csname PY@tok@nd\endcsname{\def\PY@tc##1{\textcolor[rgb]{0.67,0.13,1.00}{##1}}}
\expandafter\def\csname PY@tok@ne\endcsname{\let\PY@bf=\textbf\def\PY@tc##1{\textcolor[rgb]{0.82,0.25,0.23}{##1}}}
\expandafter\def\csname PY@tok@nf\endcsname{\def\PY@tc##1{\textcolor[rgb]{0.00,0.00,1.00}{##1}}}
\expandafter\def\csname PY@tok@si\endcsname{\let\PY@bf=\textbf\def\PY@tc##1{\textcolor[rgb]{0.73,0.40,0.53}{##1}}}
\expandafter\def\csname PY@tok@s2\endcsname{\def\PY@tc##1{\textcolor[rgb]{0.73,0.13,0.13}{##1}}}
\expandafter\def\csname PY@tok@vi\endcsname{\def\PY@tc##1{\textcolor[rgb]{0.10,0.09,0.49}{##1}}}
\expandafter\def\csname PY@tok@nt\endcsname{\let\PY@bf=\textbf\def\PY@tc##1{\textcolor[rgb]{0.00,0.50,0.00}{##1}}}
\expandafter\def\csname PY@tok@nv\endcsname{\def\PY@tc##1{\textcolor[rgb]{0.10,0.09,0.49}{##1}}}
\expandafter\def\csname PY@tok@s1\endcsname{\def\PY@tc##1{\textcolor[rgb]{0.73,0.13,0.13}{##1}}}
\expandafter\def\csname PY@tok@sh\endcsname{\def\PY@tc##1{\textcolor[rgb]{0.73,0.13,0.13}{##1}}}
\expandafter\def\csname PY@tok@sc\endcsname{\def\PY@tc##1{\textcolor[rgb]{0.73,0.13,0.13}{##1}}}
\expandafter\def\csname PY@tok@sx\endcsname{\def\PY@tc##1{\textcolor[rgb]{0.00,0.50,0.00}{##1}}}
\expandafter\def\csname PY@tok@bp\endcsname{\def\PY@tc##1{\textcolor[rgb]{0.00,0.50,0.00}{##1}}}
\expandafter\def\csname PY@tok@c1\endcsname{\let\PY@it=\textit\def\PY@tc##1{\textcolor[rgb]{0.25,0.50,0.50}{##1}}}
\expandafter\def\csname PY@tok@kc\endcsname{\let\PY@bf=\textbf\def\PY@tc##1{\textcolor[rgb]{0.00,0.50,0.00}{##1}}}
\expandafter\def\csname PY@tok@c\endcsname{\let\PY@it=\textit\def\PY@tc##1{\textcolor[rgb]{0.25,0.50,0.50}{##1}}}
\expandafter\def\csname PY@tok@mf\endcsname{\def\PY@tc##1{\textcolor[rgb]{0.40,0.40,0.40}{##1}}}
\expandafter\def\csname PY@tok@err\endcsname{\def\PY@bc##1{\setlength{\fboxsep}{0pt}\fcolorbox[rgb]{1.00,0.00,0.00}{1,1,1}{\strut ##1}}}
\expandafter\def\csname PY@tok@kd\endcsname{\let\PY@bf=\textbf\def\PY@tc##1{\textcolor[rgb]{0.00,0.50,0.00}{##1}}}
\expandafter\def\csname PY@tok@ss\endcsname{\def\PY@tc##1{\textcolor[rgb]{0.10,0.09,0.49}{##1}}}
\expandafter\def\csname PY@tok@sr\endcsname{\def\PY@tc##1{\textcolor[rgb]{0.73,0.40,0.53}{##1}}}
\expandafter\def\csname PY@tok@mo\endcsname{\def\PY@tc##1{\textcolor[rgb]{0.40,0.40,0.40}{##1}}}
\expandafter\def\csname PY@tok@kn\endcsname{\let\PY@bf=\textbf\def\PY@tc##1{\textcolor[rgb]{0.00,0.50,0.00}{##1}}}
\expandafter\def\csname PY@tok@mi\endcsname{\def\PY@tc##1{\textcolor[rgb]{0.40,0.40,0.40}{##1}}}
\expandafter\def\csname PY@tok@gp\endcsname{\let\PY@bf=\textbf\def\PY@tc##1{\textcolor[rgb]{0.00,0.00,0.50}{##1}}}
\expandafter\def\csname PY@tok@o\endcsname{\def\PY@tc##1{\textcolor[rgb]{0.40,0.40,0.40}{##1}}}
\expandafter\def\csname PY@tok@kr\endcsname{\let\PY@bf=\textbf\def\PY@tc##1{\textcolor[rgb]{0.00,0.50,0.00}{##1}}}
\expandafter\def\csname PY@tok@s\endcsname{\def\PY@tc##1{\textcolor[rgb]{0.73,0.13,0.13}{##1}}}
\expandafter\def\csname PY@tok@kp\endcsname{\def\PY@tc##1{\textcolor[rgb]{0.00,0.50,0.00}{##1}}}
\expandafter\def\csname PY@tok@w\endcsname{\def\PY@tc##1{\textcolor[rgb]{0.73,0.73,0.73}{##1}}}
\expandafter\def\csname PY@tok@kt\endcsname{\def\PY@tc##1{\textcolor[rgb]{0.69,0.00,0.25}{##1}}}
\expandafter\def\csname PY@tok@ow\endcsname{\let\PY@bf=\textbf\def\PY@tc##1{\textcolor[rgb]{0.67,0.13,1.00}{##1}}}
\expandafter\def\csname PY@tok@sb\endcsname{\def\PY@tc##1{\textcolor[rgb]{0.73,0.13,0.13}{##1}}}
\expandafter\def\csname PY@tok@k\endcsname{\let\PY@bf=\textbf\def\PY@tc##1{\textcolor[rgb]{0.00,0.50,0.00}{##1}}}
\expandafter\def\csname PY@tok@se\endcsname{\let\PY@bf=\textbf\def\PY@tc##1{\textcolor[rgb]{0.73,0.40,0.13}{##1}}}
\expandafter\def\csname PY@tok@sd\endcsname{\let\PY@it=\textit\def\PY@tc##1{\textcolor[rgb]{0.73,0.13,0.13}{##1}}}

\def\PYZbs{\char`\\}
\def\PYZus{\char`\_}
\def\PYZob{\char`\{}
\def\PYZcb{\char`\}}
\def\PYZca{\char`\^}
\def\PYZam{\char`\&}
\def\PYZlt{\char`\<}
\def\PYZgt{\char`\>}
\def\PYZsh{\char`\#}
\def\PYZpc{\char`\%}
\def\PYZdl{\char`\$}
\def\PYZhy{\char`\-}
\def\PYZsq{\char`\'}
\def\PYZdq{\char`\"}
\def\PYZti{\char`\~}
% for compatibility with earlier versions
\def\PYZat{@}
\def\PYZlb{[}
\def\PYZrb{]}
\makeatother


    %Set pygments styles if needed...
    
        \definecolor{nbframe-border}{rgb}{0.867,0.867,0.867}
        \definecolor{nbframe-bg}{rgb}{0.969,0.969,0.969}
        \definecolor{nbframe-in-prompt}{rgb}{0.0,0.0,0.502}
        \definecolor{nbframe-out-prompt}{rgb}{0.545,0.0,0.0}

        \newenvironment{ColorVerbatim}
        {\begin{mdframed}[%
            roundcorner=1.0pt, %
            backgroundcolor=nbframe-bg, %
            userdefinedwidth=1\linewidth, %
            leftmargin=0.1\linewidth, %
            innerleftmargin=0pt, %
            innerrightmargin=0pt, %
            linecolor=nbframe-border, %
            linewidth=1pt, %
            usetwoside=false, %
            everyline=true, %
            innerlinewidth=3pt, %
            innerlinecolor=nbframe-bg, %
            middlelinewidth=1pt, %
            middlelinecolor=nbframe-bg, %
            outerlinewidth=0.5pt, %
            outerlinecolor=nbframe-border, %
            needspace=0pt
        ]}
        {\end{mdframed}}
        
        \newenvironment{InvisibleVerbatim}
        {\begin{mdframed}[leftmargin=0.1\linewidth,innerleftmargin=3pt,innerrightmargin=3pt, userdefinedwidth=1\linewidth, linewidth=0pt, linecolor=white, usetwoside=false]}
        {\end{mdframed}}

        \renewenvironment{Verbatim}[1][\unskip]
        {\begin{alltt}\smaller}
        {\end{alltt}}
    

    % Help prevent overflowing lines due to urls and other hard-to-break 
    % entities.  This doesn't catch everything...
    \sloppy

    % Document level variables
    \title{pygame}
    \date{February 24, 2014}
    \release{}
    \author{Michele Pratusevich}
    \renewcommand{\releasename}{}

    % TODO: Add option for the user to specify a logo for his/her export.
    \newcommand{\sphinxlogo}{}

    % Make the index page of the document.
    \makeindex

    % Import sphinx document type specifics.
     


% Body

    % Start of the document
    \begin{document}

        
            \maketitle
        

        


        
        \section{Pygame}\begin{verbatim}
import pygame
\end{verbatim}
Pygame docs:
\href{https://www.pygame.org/docs/index.html}{https://www.pygame.org/docs/index.html}A Python \textbf{module} that is mostly used to make games, but is also
a great way to make GUIs in Python. It relies on the ideas of
object-oriented programming, classes, and objects.\subsection{A few concepts we need before using PyGame}\subsubsection{Sizes and locations}\begin{itemize}
\item
  Everything is measured in pixels. A pixel is a dot on the screen.
\item
  The entire screen is an x, y coordinate system with the TOP LEFT
  corner being 0, 0
\end{itemize}\subsubsection{Setting up the screen}\begin{verbatim}
main_screen = pygame.display.set_mode((400, 400))
main_screen.fill((255,255,255))
\end{verbatim}\subsubsection{RGB color}\begin{itemize}
\item
  Computer screen (rendered) color is described by three numbers - Red,
  Green, and Blue (RGB).
\item
  Triplet (R, G, B) corresponds to the color, where each number is from
  0 to 255.
\item
  For example: (255, 0, 0) is red; (0, 255, 0) is green; (255, 255, 255)
  is white; (0, 0, 0) is black
\end{itemize}\subsubsection{Event polling / game loop}\begin{itemize}
\item
  Instead of responding immediately when an event happens, hold a master
  ``list of events'' and based on what kind it is, respond to it.
\item
  A ``game loop'' or an ``event loop'' is used to access events.
  Basically, it is an infinite loop that keeps going for the entire time
  the game is running, and only stops when you stop the game.
\end{itemize}An example of a game loop:

\begin{verbatim}
while True: 
    ev = pygame.event.poll()
    if ev.type == pygame.QUIT: 
        sys.exit()
    if ev.type == pygame.MOUSEBUTTONDOWN: 
        x, y = ev.pos
        # do something with the click
    pygame.display.flip()
\end{verbatim}\subsubsection{Rectangles}\begin{itemize}
\item
  Everything in pygame is organized by rectangles. You draw shapes and
  buttons into rectangles, you click on rectangles, you work with
  rectangles. A rectangle represents a space where you will draw
  something.
\end{itemize}The four numbers represent the (top left x coordinate, top left y
coordinate, width, height)

\begin{verbatim}
button_rec = pygame.Rect(100, 200, 50, 20)
\end{verbatim}\subsubsection{Surface}\begin{itemize}
\item
  The type representing ``image on the screen.'' You can make it
  colorful or actually add an image to it.
\end{itemize}For an image:

\begin{verbatim}
buttonimg = pygame.image.load('baby-bamba.jpg')
\end{verbatim}\begin{itemize}
\item
  The default size will be the size of the image. There is no way to
  resize it - if you need to resize it, use a photo-editing program like
  Gimp.
\end{itemize}For a square of black color:

\begin{verbatim}
button_sq = pygame.Surface([20, 20])
\end{verbatim}
For a square of another color:

\begin{verbatim}
button_sq = pygame.Surface([20, 20])
button_sq.fill((255, 0, 0))
\end{verbatim}\subsubsection{Blit}\begin{itemize}
\item
  Think of it like ``drawing on the screen.'' What it actually means is
  ``change the pixel color inside a rectangle''
\end{itemize}\begin{verbatim}
button_rec = pygame.Rect(100, 100, 20, 20)
button_sq = pygame.Surface([20, 20])
main_screen.blit(button_sq, button_rec)
\end{verbatim}\subsubsection{Flip}\begin{itemize}
\item
  Redraw the entire window. Very costly, so you shouldn't do this very
  often.
\end{itemize}\begin{verbatim}
pygame.display.flip()
\end{verbatim}A very basic example of a Pygame program is shown here:

    % Make sure that atleast 4 lines are below the HR
    \needspace{4\baselineskip}

    
        \vspace{6pt}
        \makebox[0.1\linewidth]{\smaller\hfill\tt\color{nbframe-in-prompt}In\hspace{4pt}{[}{]}:\hspace{4pt}}\\*
        \vspace{-2.65\baselineskip}
        \begin{ColorVerbatim}
            \vspace{-0.7\baselineskip}
            \begin{Verbatim}[commandchars=\\\{\}]
\PY{k+kn}{import} \PY{n+nn}{pygame}
\PY{k+kn}{import} \PY{n+nn}{sys}

\PY{k}{if} \PY{n}{\PYZus{}\PYZus{}name\PYZus{}\PYZus{}}\PY{o}{==}\PY{l+s}{\PYZdq{}}\PY{l+s}{\PYZus{}\PYZus{}main\PYZus{}\PYZus{}}\PY{l+s}{\PYZdq{}}\PY{p}{:} 
    \PY{n}{pygame}\PY{o}{.}\PY{n}{init}\PY{p}{(}\PY{p}{)}
    \PY{n}{main\PYZus{}screen} \PY{o}{=} \PY{n}{pygame}\PY{o}{.}\PY{n}{display}\PY{o}{.}\PY{n}{set\PYZus{}mode}\PY{p}{(}\PY{p}{(}\PY{l+m+mi}{400}\PY{p}{,} \PY{l+m+mi}{400}\PY{p}{)}\PY{p}{)}
    \PY{n}{main\PYZus{}screen}\PY{o}{.}\PY{n}{fill}\PY{p}{(}\PY{p}{(}\PY{l+m+mi}{255}\PY{p}{,}\PY{l+m+mi}{255}\PY{p}{,}\PY{l+m+mi}{255}\PY{p}{)}\PY{p}{)}
    \PY{n}{button\PYZus{}rec} \PY{o}{=} \PY{n}{pygame}\PY{o}{.}\PY{n}{Rect}\PY{p}{(}\PY{l+m+mi}{100}\PY{p}{,} \PY{l+m+mi}{100}\PY{p}{,} \PY{l+m+mi}{20}\PY{p}{,} \PY{l+m+mi}{20}\PY{p}{)}
    \PY{n}{button\PYZus{}sq} \PY{o}{=} \PY{n}{pygame}\PY{o}{.}\PY{n}{Surface}\PY{p}{(}\PY{p}{[}\PY{l+m+mi}{20}\PY{p}{,} \PY{l+m+mi}{20}\PY{p}{]}\PY{p}{)}
    \PY{n}{main\PYZus{}screen}\PY{o}{.}\PY{n}{blit}\PY{p}{(}\PY{n}{button\PYZus{}sq}\PY{p}{,} \PY{n}{button\PYZus{}rec}\PY{p}{)}
    
    \PY{k}{while} \PY{n+nb+bp}{True}\PY{p}{:} 
        \PY{n}{ev} \PY{o}{=} \PY{n}{pygame}\PY{o}{.}\PY{n}{event}\PY{o}{.}\PY{n}{poll}\PY{p}{(}\PY{p}{)}
        \PY{k}{if} \PY{n}{ev}\PY{o}{.}\PY{n}{type} \PY{o}{==} \PY{n}{pygame}\PY{o}{.}\PY{n}{QUIT}\PY{p}{:} 
            \PY{n}{sys}\PY{o}{.}\PY{n}{exit}\PY{p}{(}\PY{p}{)}
        \PY{k}{if} \PY{n}{ev}\PY{o}{.}\PY{n}{type} \PY{o}{==} \PY{n}{pygame}\PY{o}{.}\PY{n}{MOUSEBUTTONDOWN}\PY{p}{:}
            \PY{n}{x}\PY{p}{,} \PY{n}{y} \PY{o}{=} \PY{n}{ev}\PY{o}{.}\PY{n}{pos}
            \PY{k}{if} \PY{n}{button\PYZus{}rec}\PY{o}{.}\PY{n}{collidepoint}\PY{p}{(}\PY{n}{x}\PY{p}{,} \PY{n}{y}\PY{p}{)}\PY{p}{:}
                \PY{k}{print} \PY{l+s}{\PYZdq{}}\PY{l+s}{you clicked me!}\PY{l+s}{\PYZdq{}}
        \PY{n}{pygame}\PY{o}{.}\PY{n}{display}\PY{o}{.}\PY{n}{flip}\PY{p}{(}\PY{p}{)}
\end{Verbatim}

            
                \vspace{-0.2\baselineskip}
            
        \end{ColorVerbatim}
    
And a commented version is here:

    % Make sure that atleast 4 lines are below the HR
    \needspace{4\baselineskip}

    
        \vspace{6pt}
        \makebox[0.1\linewidth]{\smaller\hfill\tt\color{nbframe-in-prompt}In\hspace{4pt}{[}{]}:\hspace{4pt}}\\*
        \vspace{-2.65\baselineskip}
        \begin{ColorVerbatim}
            \vspace{-0.7\baselineskip}
            \begin{Verbatim}[commandchars=\\\{\}]
\PY{k+kn}{import} \PY{n+nn}{pygame}
\PY{k+kn}{import} \PY{n+nn}{sys}

\PY{c}{\PYZsh{} custom classes here}


\PY{c}{\PYZsh{} functions not related to classes here}


\PY{c}{\PYZsh{} main method}
\PY{k}{if} \PY{n}{\PYZus{}\PYZus{}name\PYZus{}\PYZus{}}\PY{o}{==}\PY{l+s}{\PYZdq{}}\PY{l+s}{\PYZus{}\PYZus{}main\PYZus{}\PYZus{}}\PY{l+s}{\PYZdq{}}\PY{p}{:} 
    \PY{c}{\PYZsh{} start the pygame module}
    \PY{n}{pygame}\PY{o}{.}\PY{n}{init}\PY{p}{(}\PY{p}{)}
    
    \PY{c}{\PYZsh{} the two numbers represent the X and Y window size (in pixels)}
    \PY{n}{main\PYZus{}screen} \PY{o}{=} \PY{n}{pygame}\PY{o}{.}\PY{n}{display}\PY{o}{.}\PY{n}{set\PYZus{}mode}\PY{p}{(}\PY{p}{(}\PY{l+m+mi}{400}\PY{p}{,} \PY{l+m+mi}{400}\PY{p}{)}\PY{p}{)}
    
    \PY{c}{\PYZsh{} set the background color of the window \PYZhy{} remember RGB}
    \PY{n}{main\PYZus{}screen}\PY{o}{.}\PY{n}{fill}\PY{p}{(}\PY{p}{(}\PY{l+m+mi}{255}\PY{p}{,}\PY{l+m+mi}{255}\PY{p}{,}\PY{l+m+mi}{255}\PY{p}{)}\PY{p}{)}
    
    \PY{c}{\PYZsh{} make your rectangles and surfaces here}
    \PY{n}{button\PYZus{}rec} \PY{o}{=} \PY{n}{pygame}\PY{o}{.}\PY{n}{Rect}\PY{p}{(}\PY{l+m+mi}{100}\PY{p}{,} \PY{l+m+mi}{100}\PY{p}{,} \PY{l+m+mi}{20}\PY{p}{,} \PY{l+m+mi}{20}\PY{p}{)}
    \PY{n}{button\PYZus{}sq} \PY{o}{=} \PY{n}{pygame}\PY{o}{.}\PY{n}{Surface}\PY{p}{(}\PY{p}{[}\PY{l+m+mi}{20}\PY{p}{,} \PY{l+m+mi}{20}\PY{p}{]}\PY{p}{)}
    
    \PY{c}{\PYZsh{} blit the button we created at the top here, because it will not change for the rest of the game}
    \PY{c}{\PYZsh{} if it will change, blit it inside the game loop}
    \PY{n}{main\PYZus{}screen}\PY{o}{.}\PY{n}{blit}\PY{p}{(}\PY{n}{button\PYZus{}sq}\PY{p}{,} \PY{n}{button\PYZus{}rec}\PY{p}{)}
    
    \PY{c}{\PYZsh{} game / event / main loop}
    \PY{k}{while} \PY{n+nb+bp}{True}\PY{p}{:} 
        \PY{c}{\PYZsh{} get the last event that happened}
        \PY{n}{ev} \PY{o}{=} \PY{n}{pygame}\PY{o}{.}\PY{n}{event}\PY{o}{.}\PY{n}{poll}\PY{p}{(}\PY{p}{)}
        
        \PY{c}{\PYZsh{} if you clicked the X button to exit the window}
        \PY{k}{if} \PY{n}{ev}\PY{o}{.}\PY{n}{type} \PY{o}{==} \PY{n}{pygame}\PY{o}{.}\PY{n}{QUIT}\PY{p}{:} 
            
            \PY{c}{\PYZsh{} close the window}
            \PY{n}{sys}\PY{o}{.}\PY{n}{exit}\PY{p}{(}\PY{p}{)}
        
        \PY{c}{\PYZsh{} if you clicked down with the mouse}
        \PY{k}{if} \PY{n}{ev}\PY{o}{.}\PY{n}{type} \PY{o}{==} \PY{n}{pygame}\PY{o}{.}\PY{n}{MOUSEBUTTONDOWN}\PY{p}{:}
            \PY{c}{\PYZsh{} x, y represents the position of the click in the screen}
            \PY{n}{x}\PY{p}{,} \PY{n}{y} \PY{o}{=} \PY{n}{ev}\PY{o}{.}\PY{n}{pos}
            
            \PY{c}{\PYZsh{} do something with the click}
            
            \PY{c}{\PYZsh{} check if the position of the click is inside the button}
            \PY{k}{if} \PY{n}{button\PYZus{}rec}\PY{o}{.}\PY{n}{collidepoint}\PY{p}{(}\PY{n}{x}\PY{p}{,} \PY{n}{y}\PY{p}{)}\PY{p}{:}
                \PY{k}{print} \PY{l+s}{\PYZdq{}}\PY{l+s}{you clicked me!}\PY{l+s}{\PYZdq{}}
            
        \PY{c}{\PYZsh{} every game loop, make sure you update the display, but only ONCE AT THE END!}
        \PY{n}{pygame}\PY{o}{.}\PY{n}{display}\PY{o}{.}\PY{n}{flip}\PY{p}{(}\PY{p}{)}
\end{Verbatim}

            
                \vspace{-0.2\baselineskip}
            
        \end{ColorVerbatim}
    
\subsection{More things}\subsubsection{Making a text label}\begin{verbatim}
label_rec = pygame.Rect(50, 50, 200, 30)

# first argument is a filename (none if you want the default)
# second argument is the font size
orderlabel = pygame.font.Font(None, 30)

# Method: render(text, antialias, color, background=None)
label = orderlabel.render("Pillow", 1, (0, 0, 0), (255, 255, 255))

main_screen.blit(label, label_rec)
\end{verbatim}\subsubsection{Screens}\begin{itemize}
\item
  You can only have one window at a time in pygame, so everything you do
  will have to be on the same screen - no popup windows!
\end{itemize}
        

        \renewcommand{\indexname}{Index}
        \printindex

    % End of document
    \end{document}


