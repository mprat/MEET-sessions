


% Header, overrides base

    % Make sure that the sphinx doc style knows who it inherits from.
    \def\sphinxdocclass{article}

    % Declare the document class
    \documentclass[letterpaper,10pt,english]{/usr/local/lib/python2.7/dist-packages/sphinx/texinputs/sphinxhowto}

    % Imports
    \usepackage[utf8]{inputenc}
    \DeclareUnicodeCharacter{00A0}{\\nobreakspace}
    \usepackage[T1]{fontenc}
    \usepackage{babel}
    \usepackage{times}
    \usepackage{import}
    \usepackage[Bjarne]{/usr/local/lib/python2.7/dist-packages/sphinx/texinputs/fncychap}
    \usepackage{longtable}
    \usepackage{/usr/local/lib/python2.7/dist-packages/sphinx/texinputs/sphinx}
    \usepackage{multirow}

    \usepackage{amsmath}
    \usepackage{amssymb}
    \usepackage{ucs}
    \usepackage{enumerate}

    % Used to make the Input/Output rules follow around the contents.
    \usepackage{needspace}

    % Pygments requirements
    \usepackage{fancyvrb}
    \usepackage{color}
    % ansi colors additions
    \definecolor{darkgreen}{rgb}{.12,.54,.11}
    \definecolor{lightgray}{gray}{.95}
    \definecolor{brown}{rgb}{0.54,0.27,0.07}
    \definecolor{purple}{rgb}{0.5,0.0,0.5}
    \definecolor{darkgray}{gray}{0.25}
    \definecolor{lightred}{rgb}{1.0,0.39,0.28}
    \definecolor{lightgreen}{rgb}{0.48,0.99,0.0}
    \definecolor{lightblue}{rgb}{0.53,0.81,0.92}
    \definecolor{lightpurple}{rgb}{0.87,0.63,0.87}
    \definecolor{lightcyan}{rgb}{0.5,1.0,0.83}

    % Needed to box output/input
    \usepackage{tikz}
        \usetikzlibrary{calc,arrows,shadows}
    \usepackage[framemethod=tikz]{mdframed}

    \usepackage{alltt}

    % Used to load and display graphics
    \usepackage{graphicx}
    \graphicspath{ {figs/} }
    \usepackage[Export]{adjustbox} % To resize

    % used so that images for notebooks which have spaces in the name can still be included
    \usepackage{grffile}


    % For formatting output while also word wrapping.
    \usepackage{listings}
    \lstset{breaklines=true}
    \lstset{basicstyle=\small\ttfamily}
    \def\smaller{\fontsize{9.5pt}{9.5pt}\selectfont}

    %Pygments definitions
    
\makeatletter
\def\PY@reset{\let\PY@it=\relax \let\PY@bf=\relax%
    \let\PY@ul=\relax \let\PY@tc=\relax%
    \let\PY@bc=\relax \let\PY@ff=\relax}
\def\PY@tok#1{\csname PY@tok@#1\endcsname}
\def\PY@toks#1+{\ifx\relax#1\empty\else%
    \PY@tok{#1}\expandafter\PY@toks\fi}
\def\PY@do#1{\PY@bc{\PY@tc{\PY@ul{%
    \PY@it{\PY@bf{\PY@ff{#1}}}}}}}
\def\PY#1#2{\PY@reset\PY@toks#1+\relax+\PY@do{#2}}

\expandafter\def\csname PY@tok@gd\endcsname{\def\PY@tc##1{\textcolor[rgb]{0.63,0.00,0.00}{##1}}}
\expandafter\def\csname PY@tok@gu\endcsname{\let\PY@bf=\textbf\def\PY@tc##1{\textcolor[rgb]{0.50,0.00,0.50}{##1}}}
\expandafter\def\csname PY@tok@gt\endcsname{\def\PY@tc##1{\textcolor[rgb]{0.00,0.27,0.87}{##1}}}
\expandafter\def\csname PY@tok@gs\endcsname{\let\PY@bf=\textbf}
\expandafter\def\csname PY@tok@gr\endcsname{\def\PY@tc##1{\textcolor[rgb]{1.00,0.00,0.00}{##1}}}
\expandafter\def\csname PY@tok@cm\endcsname{\let\PY@it=\textit\def\PY@tc##1{\textcolor[rgb]{0.25,0.50,0.50}{##1}}}
\expandafter\def\csname PY@tok@vg\endcsname{\def\PY@tc##1{\textcolor[rgb]{0.10,0.09,0.49}{##1}}}
\expandafter\def\csname PY@tok@m\endcsname{\def\PY@tc##1{\textcolor[rgb]{0.40,0.40,0.40}{##1}}}
\expandafter\def\csname PY@tok@mh\endcsname{\def\PY@tc##1{\textcolor[rgb]{0.40,0.40,0.40}{##1}}}
\expandafter\def\csname PY@tok@go\endcsname{\def\PY@tc##1{\textcolor[rgb]{0.53,0.53,0.53}{##1}}}
\expandafter\def\csname PY@tok@ge\endcsname{\let\PY@it=\textit}
\expandafter\def\csname PY@tok@vc\endcsname{\def\PY@tc##1{\textcolor[rgb]{0.10,0.09,0.49}{##1}}}
\expandafter\def\csname PY@tok@il\endcsname{\def\PY@tc##1{\textcolor[rgb]{0.40,0.40,0.40}{##1}}}
\expandafter\def\csname PY@tok@cs\endcsname{\let\PY@it=\textit\def\PY@tc##1{\textcolor[rgb]{0.25,0.50,0.50}{##1}}}
\expandafter\def\csname PY@tok@cp\endcsname{\def\PY@tc##1{\textcolor[rgb]{0.74,0.48,0.00}{##1}}}
\expandafter\def\csname PY@tok@gi\endcsname{\def\PY@tc##1{\textcolor[rgb]{0.00,0.63,0.00}{##1}}}
\expandafter\def\csname PY@tok@gh\endcsname{\let\PY@bf=\textbf\def\PY@tc##1{\textcolor[rgb]{0.00,0.00,0.50}{##1}}}
\expandafter\def\csname PY@tok@ni\endcsname{\let\PY@bf=\textbf\def\PY@tc##1{\textcolor[rgb]{0.60,0.60,0.60}{##1}}}
\expandafter\def\csname PY@tok@nl\endcsname{\def\PY@tc##1{\textcolor[rgb]{0.63,0.63,0.00}{##1}}}
\expandafter\def\csname PY@tok@nn\endcsname{\let\PY@bf=\textbf\def\PY@tc##1{\textcolor[rgb]{0.00,0.00,1.00}{##1}}}
\expandafter\def\csname PY@tok@no\endcsname{\def\PY@tc##1{\textcolor[rgb]{0.53,0.00,0.00}{##1}}}
\expandafter\def\csname PY@tok@na\endcsname{\def\PY@tc##1{\textcolor[rgb]{0.49,0.56,0.16}{##1}}}
\expandafter\def\csname PY@tok@nb\endcsname{\def\PY@tc##1{\textcolor[rgb]{0.00,0.50,0.00}{##1}}}
\expandafter\def\csname PY@tok@nc\endcsname{\let\PY@bf=\textbf\def\PY@tc##1{\textcolor[rgb]{0.00,0.00,1.00}{##1}}}
\expandafter\def\csname PY@tok@nd\endcsname{\def\PY@tc##1{\textcolor[rgb]{0.67,0.13,1.00}{##1}}}
\expandafter\def\csname PY@tok@ne\endcsname{\let\PY@bf=\textbf\def\PY@tc##1{\textcolor[rgb]{0.82,0.25,0.23}{##1}}}
\expandafter\def\csname PY@tok@nf\endcsname{\def\PY@tc##1{\textcolor[rgb]{0.00,0.00,1.00}{##1}}}
\expandafter\def\csname PY@tok@si\endcsname{\let\PY@bf=\textbf\def\PY@tc##1{\textcolor[rgb]{0.73,0.40,0.53}{##1}}}
\expandafter\def\csname PY@tok@s2\endcsname{\def\PY@tc##1{\textcolor[rgb]{0.73,0.13,0.13}{##1}}}
\expandafter\def\csname PY@tok@vi\endcsname{\def\PY@tc##1{\textcolor[rgb]{0.10,0.09,0.49}{##1}}}
\expandafter\def\csname PY@tok@nt\endcsname{\let\PY@bf=\textbf\def\PY@tc##1{\textcolor[rgb]{0.00,0.50,0.00}{##1}}}
\expandafter\def\csname PY@tok@nv\endcsname{\def\PY@tc##1{\textcolor[rgb]{0.10,0.09,0.49}{##1}}}
\expandafter\def\csname PY@tok@s1\endcsname{\def\PY@tc##1{\textcolor[rgb]{0.73,0.13,0.13}{##1}}}
\expandafter\def\csname PY@tok@sh\endcsname{\def\PY@tc##1{\textcolor[rgb]{0.73,0.13,0.13}{##1}}}
\expandafter\def\csname PY@tok@sc\endcsname{\def\PY@tc##1{\textcolor[rgb]{0.73,0.13,0.13}{##1}}}
\expandafter\def\csname PY@tok@sx\endcsname{\def\PY@tc##1{\textcolor[rgb]{0.00,0.50,0.00}{##1}}}
\expandafter\def\csname PY@tok@bp\endcsname{\def\PY@tc##1{\textcolor[rgb]{0.00,0.50,0.00}{##1}}}
\expandafter\def\csname PY@tok@c1\endcsname{\let\PY@it=\textit\def\PY@tc##1{\textcolor[rgb]{0.25,0.50,0.50}{##1}}}
\expandafter\def\csname PY@tok@kc\endcsname{\let\PY@bf=\textbf\def\PY@tc##1{\textcolor[rgb]{0.00,0.50,0.00}{##1}}}
\expandafter\def\csname PY@tok@c\endcsname{\let\PY@it=\textit\def\PY@tc##1{\textcolor[rgb]{0.25,0.50,0.50}{##1}}}
\expandafter\def\csname PY@tok@mf\endcsname{\def\PY@tc##1{\textcolor[rgb]{0.40,0.40,0.40}{##1}}}
\expandafter\def\csname PY@tok@err\endcsname{\def\PY@bc##1{\setlength{\fboxsep}{0pt}\fcolorbox[rgb]{1.00,0.00,0.00}{1,1,1}{\strut ##1}}}
\expandafter\def\csname PY@tok@kd\endcsname{\let\PY@bf=\textbf\def\PY@tc##1{\textcolor[rgb]{0.00,0.50,0.00}{##1}}}
\expandafter\def\csname PY@tok@ss\endcsname{\def\PY@tc##1{\textcolor[rgb]{0.10,0.09,0.49}{##1}}}
\expandafter\def\csname PY@tok@sr\endcsname{\def\PY@tc##1{\textcolor[rgb]{0.73,0.40,0.53}{##1}}}
\expandafter\def\csname PY@tok@mo\endcsname{\def\PY@tc##1{\textcolor[rgb]{0.40,0.40,0.40}{##1}}}
\expandafter\def\csname PY@tok@kn\endcsname{\let\PY@bf=\textbf\def\PY@tc##1{\textcolor[rgb]{0.00,0.50,0.00}{##1}}}
\expandafter\def\csname PY@tok@mi\endcsname{\def\PY@tc##1{\textcolor[rgb]{0.40,0.40,0.40}{##1}}}
\expandafter\def\csname PY@tok@gp\endcsname{\let\PY@bf=\textbf\def\PY@tc##1{\textcolor[rgb]{0.00,0.00,0.50}{##1}}}
\expandafter\def\csname PY@tok@o\endcsname{\def\PY@tc##1{\textcolor[rgb]{0.40,0.40,0.40}{##1}}}
\expandafter\def\csname PY@tok@kr\endcsname{\let\PY@bf=\textbf\def\PY@tc##1{\textcolor[rgb]{0.00,0.50,0.00}{##1}}}
\expandafter\def\csname PY@tok@s\endcsname{\def\PY@tc##1{\textcolor[rgb]{0.73,0.13,0.13}{##1}}}
\expandafter\def\csname PY@tok@kp\endcsname{\def\PY@tc##1{\textcolor[rgb]{0.00,0.50,0.00}{##1}}}
\expandafter\def\csname PY@tok@w\endcsname{\def\PY@tc##1{\textcolor[rgb]{0.73,0.73,0.73}{##1}}}
\expandafter\def\csname PY@tok@kt\endcsname{\def\PY@tc##1{\textcolor[rgb]{0.69,0.00,0.25}{##1}}}
\expandafter\def\csname PY@tok@ow\endcsname{\let\PY@bf=\textbf\def\PY@tc##1{\textcolor[rgb]{0.67,0.13,1.00}{##1}}}
\expandafter\def\csname PY@tok@sb\endcsname{\def\PY@tc##1{\textcolor[rgb]{0.73,0.13,0.13}{##1}}}
\expandafter\def\csname PY@tok@k\endcsname{\let\PY@bf=\textbf\def\PY@tc##1{\textcolor[rgb]{0.00,0.50,0.00}{##1}}}
\expandafter\def\csname PY@tok@se\endcsname{\let\PY@bf=\textbf\def\PY@tc##1{\textcolor[rgb]{0.73,0.40,0.13}{##1}}}
\expandafter\def\csname PY@tok@sd\endcsname{\let\PY@it=\textit\def\PY@tc##1{\textcolor[rgb]{0.73,0.13,0.13}{##1}}}

\def\PYZbs{\char`\\}
\def\PYZus{\char`\_}
\def\PYZob{\char`\{}
\def\PYZcb{\char`\}}
\def\PYZca{\char`\^}
\def\PYZam{\char`\&}
\def\PYZlt{\char`\<}
\def\PYZgt{\char`\>}
\def\PYZsh{\char`\#}
\def\PYZpc{\char`\%}
\def\PYZdl{\char`\$}
\def\PYZhy{\char`\-}
\def\PYZsq{\char`\'}
\def\PYZdq{\char`\"}
\def\PYZti{\char`\~}
% for compatibility with earlier versions
\def\PYZat{@}
\def\PYZlb{[}
\def\PYZrb{]}
\makeatother


    %Set pygments styles if needed...
    
        \definecolor{nbframe-border}{rgb}{0.867,0.867,0.867}
        \definecolor{nbframe-bg}{rgb}{0.969,0.969,0.969}
        \definecolor{nbframe-in-prompt}{rgb}{0.0,0.0,0.502}
        \definecolor{nbframe-out-prompt}{rgb}{0.545,0.0,0.0}

        \newenvironment{ColorVerbatim}
        {\begin{mdframed}[%
            roundcorner=1.0pt, %
            backgroundcolor=nbframe-bg, %
            userdefinedwidth=1\linewidth, %
            leftmargin=0.1\linewidth, %
            innerleftmargin=0pt, %
            innerrightmargin=0pt, %
            linecolor=nbframe-border, %
            linewidth=1pt, %
            usetwoside=false, %
            everyline=true, %
            innerlinewidth=3pt, %
            innerlinecolor=nbframe-bg, %
            middlelinewidth=1pt, %
            middlelinecolor=nbframe-bg, %
            outerlinewidth=0.5pt, %
            outerlinecolor=nbframe-border, %
            needspace=0pt
        ]}
        {\end{mdframed}}
        
        \newenvironment{InvisibleVerbatim}
        {\begin{mdframed}[leftmargin=0.1\linewidth,innerleftmargin=3pt,innerrightmargin=3pt, userdefinedwidth=1\linewidth, linewidth=0pt, linecolor=white, usetwoside=false]}
        {\end{mdframed}}

        \renewenvironment{Verbatim}[1][\unskip]
        {\begin{alltt}\smaller}
        {\end{alltt}}
    

    % Help prevent overflowing lines due to urls and other hard-to-break 
    % entities.  This doesn't catch everything...
    \sloppy

    % Document level variables
    \title{OOP-lab2}
    \date{January 27, 2014}
    \release{}
    \author{Michele Pratusevich}
    \renewcommand{\releasename}{}

    % TODO: Add option for the user to specify a logo for his/her export.
    \newcommand{\sphinxlogo}{}

    % Make the index page of the document.
    \makeindex

    % Import sphinx document type specifics.
     


% Body

    % Start of the document
    \begin{document}

        
            \maketitle
        

        


        
        \section{Lab 4: OOP, classes, and objects}

\begin{enumerate}[1.]
\item
  Make a new folder called Lab4 (in your MEET-YL1 folder) to store all
  of your work. If you do not have a MEET-YL1 folder on your desktop (or
  are logging in as Guest), \texttt{cd} onto the Desktop and
  \texttt{git clone http://github.com/YOURUSERNAME13-meet/MEET-YL1} to
  get your MEET-YL1 folder.
\item
  Open an empty text file and save it as \texttt{number.py} in the Lab4
  folder.
\item
  In the \texttt{number.py} file, create a class called
  \texttt{Integer}. This class should have one variable that stores the
  number, and one method called \texttt{display} that just prints out
  the number.
\item
  Below your class definition, make a \texttt{\_\_main\_\_} method. It
  should look like this:

\begin{verbatim}
if __name__=="__main__":
    print "Michele"
\end{verbatim}
  Inside this main method, create a new \texttt{Integer} called
  \texttt{test} of the class that you created, and prove that the
  \texttt{display} method works by calling \texttt{test.display()}. This
  piece of code will serve as your tests for the classes you create.
\item
  Add a new parameter (variable) to your \texttt{Integer} class that
  will describe whether the integer is a positive or a negative number.
  (What kind of information will this variable store?). You will need to
  change your contructor to include this second variable. Change the
  \texttt{display} method to display a number correctly, based on
  whether it is positive or negative. For example, instead of displaying
  ``9'', display ``-9'' if necessary. Make sure your main method tests
  still work correctly.
\item
  Create a subclass of your \texttt{Integer} class called
  \texttt{NegativeInteger}. The constructor for this class should only
  have one parameter - asking for the number. Because the class is
  called \texttt{NegativeInteger}, one of the variables from
  \texttt{Integer} will already be set. Make sure to take care of this
  in the constructor.
\item
  In your main method, create a new object of type
  \texttt{NegativeInteger} and call it's \texttt{display} method, making
  sure it works correctly.
\item
  Override the \texttt{display} method of the \texttt{NegativeInteger}
  class. It should be the SAME as the \texttt{display} method of the
  \texttt{Integer} class, but afterwards, it should also print to the
  screen ``This is an object of the NegativeInteger class''. Remember,
  to call a method from a superclass inside a subclass, do:

\begin{verbatim}
SuperClassName.methodname(self, arguments)
\end{verbatim}
  Test to make sure these display methods work correctly.
\item
  Now instead of calling the \texttt{display} method separately for all
  the objects you created, create an \texttt{Integer} object and a
  \texttt{NegativeInteger} object. Put these objects in a list. Make a
  loop that iterates over all the objects in that list and calls their
  \texttt{display} methods.
\item
  (extra) Instead of hard-coding the numbers that you want to create in
  your main method, ask the user to give you a positive number and a
  negative number to create. (Which class types will you create from
  these objects? Does it matter?)
\item
  (extra) Ask the user how many numbers they want to create, and for
  each number, ask them whether it is positive or negative, then create
  these numbers.
\item
  (extra) Figure out a way to do arithmetic (adding, subtracting,
  multiplying, dividing) with the \texttt{Integer}s and
  \texttt{NegativeInteger}s. (\emph{Hint: One way of doing this is to
  make four new methods to the \texttt{Integer} class, called something
  like \texttt{add()}, \texttt{subtract()}, \texttt{multiply()}, and
  \texttt{divide()}. Start with addition, and figure out how to test it
  and work up from there.})
\end{enumerate}For reference, here is the example \texttt{Student} and
\texttt{UniversityStudent} class I made for you in class.

    % Make sure that atleast 4 lines are below the HR
    \needspace{4\baselineskip}

    
        \vspace{6pt}
        \makebox[0.1\linewidth]{\smaller\hfill\tt\color{nbframe-in-prompt}In\hspace{4pt}{[}1{]}:\hspace{4pt}}\\*
        \vspace{-2.65\baselineskip}
        \begin{ColorVerbatim}
            \vspace{-0.7\baselineskip}
            \begin{Verbatim}[commandchars=\\\{\}]
\PY{k}{class} \PY{n+nc}{Student}\PY{p}{(}\PY{n+nb}{object}\PY{p}{)}\PY{p}{:}
    \PY{k}{def} \PY{n+nf}{\PYZus{}\PYZus{}init\PYZus{}\PYZus{}}\PY{p}{(}\PY{n+nb+bp}{self}\PY{p}{,} \PY{n}{name\PYZus{}to\PYZus{}set}\PY{o}{=}\PY{l+s}{\PYZdq{}}\PY{l+s}{\PYZdq{}}\PY{p}{,} \PY{n}{year\PYZus{}to\PYZus{}set}\PY{o}{=}\PY{l+m+mi}{0}\PY{p}{,} \PY{n}{score}\PY{o}{=}\PY{l+m+mi}{100}\PY{p}{)}\PY{p}{:}
        \PY{n+nb+bp}{self}\PY{o}{.}\PY{n}{name} \PY{o}{=} \PY{n}{name\PYZus{}to\PYZus{}set}
        \PY{n+nb+bp}{self}\PY{o}{.}\PY{n}{year} \PY{o}{=} \PY{n}{year\PYZus{}to\PYZus{}set}
        \PY{n+nb+bp}{self}\PY{o}{.}\PY{n}{score} \PY{o}{=} \PY{n}{score}
    
    \PY{k}{def} \PY{n+nf}{letter\PYZus{}grade}\PY{p}{(}\PY{n+nb+bp}{self}\PY{p}{)}\PY{p}{:}
        \PY{k}{if} \PY{n+nb+bp}{self}\PY{o}{.}\PY{n}{score} \PY{o}{\PYZgt{}}\PY{o}{=} \PY{l+m+mi}{90}\PY{p}{:}
            \PY{k}{return} \PY{l+s}{\PYZdq{}}\PY{l+s}{A}\PY{l+s}{\PYZdq{}}
        \PY{k}{elif} \PY{n+nb+bp}{self}\PY{o}{.}\PY{n}{score} \PY{o}{\PYZgt{}}\PY{o}{=} \PY{l+m+mi}{80}\PY{p}{:} 
            \PY{k}{return} \PY{l+s}{\PYZdq{}}\PY{l+s}{B}\PY{l+s}{\PYZdq{}}
        \PY{k}{elif} \PY{n+nb+bp}{self}\PY{o}{.}\PY{n}{score} \PY{o}{\PYZgt{}}\PY{o}{=} \PY{l+m+mi}{70}\PY{p}{:}
            \PY{k}{return} \PY{l+s}{\PYZdq{}}\PY{l+s}{C}\PY{l+s}{\PYZdq{}}
        \PY{k}{elif} \PY{n+nb+bp}{self}\PY{o}{.}\PY{n}{score} \PY{o}{\PYZgt{}}\PY{o}{=} \PY{l+m+mi}{65}\PY{p}{:}
            \PY{k}{return} \PY{l+s}{\PYZdq{}}\PY{l+s}{D}\PY{l+s}{\PYZdq{}}
        \PY{k}{else}\PY{p}{:} 
            \PY{k}{return} \PY{l+s}{\PYZdq{}}\PY{l+s}{F}\PY{l+s}{\PYZdq{}}
\end{Verbatim}

            
                \vspace{-0.2\baselineskip}
            
        \end{ColorVerbatim}
    


    % Make sure that atleast 4 lines are below the HR
    \needspace{4\baselineskip}

    
        \vspace{6pt}
        \makebox[0.1\linewidth]{\smaller\hfill\tt\color{nbframe-in-prompt}In\hspace{4pt}{[}2{]}:\hspace{4pt}}\\*
        \vspace{-2.65\baselineskip}
        \begin{ColorVerbatim}
            \vspace{-0.7\baselineskip}
            \begin{Verbatim}[commandchars=\\\{\}]
\PY{k}{class} \PY{n+nc}{UniversityStudent}\PY{p}{(}\PY{n}{Student}\PY{p}{)}\PY{p}{:}
    \PY{k}{def} \PY{n+nf}{\PYZus{}\PYZus{}init\PYZus{}\PYZus{}}\PY{p}{(}\PY{n+nb+bp}{self}\PY{p}{,} \PY{n}{name\PYZus{}to\PYZus{}set}\PY{o}{=}\PY{l+s}{\PYZdq{}}\PY{l+s}{\PYZdq{}}\PY{p}{,} \PY{n}{year\PYZus{}to\PYZus{}set}\PY{o}{=}\PY{l+s}{\PYZdq{}}\PY{l+s}{\PYZdq{}}\PY{p}{,} \PY{n}{uni}\PY{o}{=}\PY{l+s}{\PYZdq{}}\PY{l+s}{\PYZdq{}}\PY{p}{)}\PY{p}{:}
        \PY{n+nb}{super}\PY{p}{(}\PY{n}{UniversityStudent}\PY{p}{,} \PY{n+nb+bp}{self}\PY{p}{)}\PY{o}{.}\PY{n}{\PYZus{}\PYZus{}init\PYZus{}\PYZus{}}\PY{p}{(}\PY{n}{name\PYZus{}to\PYZus{}set}\PY{p}{,} \PY{n}{year\PYZus{}to\PYZus{}set}\PY{p}{)}
        \PY{n+nb+bp}{self}\PY{o}{.}\PY{n}{university} \PY{o}{=} \PY{n}{uni}
    
    \PY{k}{def} \PY{n+nf}{signature}\PY{p}{(}\PY{n+nb+bp}{self}\PY{p}{)}\PY{p}{:}
        \PY{k}{return} \PY{n+nb+bp}{self}\PY{o}{.}\PY{n}{name} \PY{o}{+} \PY{l+s}{\PYZdq{}}\PY{l+s}{, }\PY{l+s}{\PYZdq{}} \PY{o}{+} \PY{n+nb+bp}{self}\PY{o}{.}\PY{n}{university} \PY{o}{+} \PY{l+s}{\PYZdq{}}\PY{l+s}{ }\PY{l+s}{\PYZdq{}} \PY{o}{+} \PY{n+nb}{str}\PY{p}{(}\PY{n+nb+bp}{self}\PY{o}{.}\PY{n}{year}\PY{p}{)} \PY{o}{+} \PY{l+s}{\PYZdq{}}\PY{l+s}{ Grade: }\PY{l+s}{\PYZdq{}} \PY{o}{+} \PY{n}{Student}\PY{o}{.}\PY{n}{letter\PYZus{}grade}\PY{p}{(}\PY{n+nb+bp}{self}\PY{p}{)}
\end{Verbatim}

            
                \vspace{-0.2\baselineskip}
            
        \end{ColorVerbatim}
    


    % Make sure that atleast 4 lines are below the HR
    \needspace{4\baselineskip}

    
        \vspace{6pt}
        \makebox[0.1\linewidth]{\smaller\hfill\tt\color{nbframe-in-prompt}In\hspace{4pt}{[}5{]}:\hspace{4pt}}\\*
        \vspace{-2.65\baselineskip}
        \begin{ColorVerbatim}
            \vspace{-0.7\baselineskip}
            \begin{Verbatim}[commandchars=\\\{\}]
\PY{k}{if} \PY{n}{\PYZus{}\PYZus{}name\PYZus{}\PYZus{}}\PY{o}{==}\PY{l+s}{\PYZdq{}}\PY{l+s}{\PYZus{}\PYZus{}main\PYZus{}\PYZus{}}\PY{l+s}{\PYZdq{}}\PY{p}{:}
    \PY{n}{m} \PY{o}{=} \PY{n}{Student}\PY{p}{(}\PY{l+s}{\PYZdq{}}\PY{l+s}{Michele Pratusevich}\PY{l+s}{\PYZdq{}}\PY{p}{,} \PY{l+m+mi}{2013}\PY{p}{)}
    \PY{n}{m}\PY{o}{.}\PY{n}{score} \PY{o}{=} \PY{l+m+mi}{85}
    \PY{k}{print} \PY{n}{m}\PY{o}{.}\PY{n}{score}
    \PY{k}{print} \PY{n}{m}\PY{o}{.}\PY{n}{letter\PYZus{}grade}\PY{p}{(}\PY{p}{)}
    
    \PY{n}{n} \PY{o}{=} \PY{n}{UniversityStudent}\PY{p}{(}\PY{l+s}{\PYZdq{}}\PY{l+s}{Kyle Hannon}\PY{l+s}{\PYZdq{}}\PY{p}{,} \PY{l+m+mi}{2013}\PY{p}{,} \PY{l+s}{\PYZdq{}}\PY{l+s}{MIT}\PY{l+s}{\PYZdq{}}\PY{p}{)}
    \PY{k}{print} \PY{n}{n}\PY{o}{.}\PY{n}{name}
    \PY{k}{print} \PY{n}{n}\PY{o}{.}\PY{n}{score}
    \PY{k}{print} \PY{n}{n}\PY{o}{.}\PY{n}{year}
    \PY{k}{print} \PY{n}{n}\PY{o}{.}\PY{n}{signature}\PY{p}{(}\PY{p}{)}
\end{Verbatim}

            
                \vspace{-0.2\baselineskip}
            
        \end{ColorVerbatim}
    

    

        % If the first block is an image, minipage the image.  Else
        % request a certain amount of space for the input text.
        \needspace{4\baselineskip}
        
        

            % Add document contents.
            
                \begin{InvisibleVerbatim}
                \vspace{-0.5\baselineskip}
\begin{alltt}85
B
Kyle Hannon
100
2013
Kyle Hannon, MIT 2013 Grade: A
\end{alltt}

            \end{InvisibleVerbatim}
            
        
    

        

        \renewcommand{\indexname}{Index}
        \printindex

    % End of document
    \end{document}


