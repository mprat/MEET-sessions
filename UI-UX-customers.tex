(intro to class) We are going to do something that is not usually done
in entrepreneurship teaching, and that is merge CS and business for the
customer exploration part of the project. The goal of today is not to
find solutions to your area / field directly, but to get at your
solution from the perspective of the customer / consumer and UI / UX.

Clarification on the use of the word ``customer'' - what we mean is
customer / stakeholder / user. The person who is being targeted with
your solution. Customer is a business term, but we don't only mean
customer in terms of business.

(15 minutes for activity) Finalize your problem / customer segments. The
goal is to figure out in general who your potential customers are - if
there are multiple potential audiences, great. Just keep in mind that
today we will be working very strongly on customers.

Goal for the end of today: really truly understand your potential
customer segments.

(3 minutes per group) Share your areas / fields / arenas; major
problems; potential customer segments (does not have to be very
detailed).

\subsection{UI vs.~UX}

INTERFACE and EXPERIENCE

\begin{itemize}
\item
  interface and the experience of a product / solution make or break the
  product
\item
  sometimes the experience IS the product / solution
\item
  Filled with examples, and I want to give you things to think about
\end{itemize}
\begin{itemize}
\item
  GetTaxi experience is different than the hailing a cab on the street
  experience
\item
  This product is not about the interface, it is about the experience
\end{itemize}
\subsection{Principles of UI / UX}

\begin{enumerate}[1.]
\item
  \textbf{Learnability} - is it easy to use?
\item
  \textbf{Efficiency} - once you learn it, is it fast to use?
\item
  \textbf{Safety} - are errors few and recoverable?
\item
  \textbf{Aesthetics} - satisfaction / happiness of user?
\item
  \textbf{Ergonomics} - is it comfortable / exhausting to use?
\end{enumerate}
\begin{itemize}
\item
  ``textbook'' UI / UX principles. There are more, but they can be
  broken into these 5 categories
\item
  Which ones you care about matter on the application. For example:
  spaceshuttle launch, subway turnstile, experts, novices
\end{itemize}
\subsection{Techniques}

\begin{itemize}
\item
  These are by no means comprehensive techniques. This is more to serve
  examples, and there are a million ways to do something, and you can
  always invent your own. These are just to give you things to think
  about
\item
  Metaphors (but don't force it)
\item
  What people know is easy for them to understand
\item
  Makes it easy to learn
\end{itemize}
\begin{itemize}
\item
  Consistency
\item
  Stay close to what people know
\item
  related to metaphors
\item
  If you are doing something radical, your UI has to at least be
  something close to what people know
\item
  The same applies for bootstrapping your ideas to something else. For
  example, if your idea is to make a game to play with others, maybe a
  Facebook game is easier for people to use than remembering to go to
  another new social networking website
\end{itemize}
\begin{itemize}
\item
  ``Affordance''
\item
  Give your users clues about what is happening (this is not related to
  the age or demographic of your users at all. this is just good for
  everyone)
\end{itemize}
\begin{itemize}
\item
  User feedback. Something should change in 100ms, or the user will
  notice that nothing happened and get frustrated
\item
  Angry users are not good users
\end{itemize}
\begin{itemize}
\item
  Direct mapping to something physical or known
\item
  related to metaphors, but more direct
\item
  People who are used to using something in the physical world will want
  to use this more
\end{itemize}
\begin{itemize}
\item
  Gamification
\item
  Make it fun
\item
  especially important for younger users, or users who have a lot of
  things thrown at them
\item
  Can help you stand out
\item
  Can help with users coming back
\item
  Competition is good too
\item
  This can be a whole separate lecture, but basically make it fun
\end{itemize}
\begin{itemize}
\item
  Look for your own examples of good UI / UX
\item
  Don't mimic, take what is good and build on it
\end{itemize}
\subsection{Design process}

\begin{itemize}
\item
  What kinds of users do you have?
  \begin{itemize}
  \item
    computer experience, age, motivation, goals, education, language,
    age?
  \item
    YOU are not the user (in most cases)
  \end{itemize}
\item
  Prototype early, prototype often
\item
  Paper prototypes
\item
  A / B testing
\end{itemize}
\subsection{How do you know what the user really wants?}

\begin{itemize}
\item
  Telephone handsets were heavy and users were asked ``what do you want
  the handset to weight'' but when the user testing was done, it turned
  out that they actually wanted different from what they said
\end{itemize}
\subsection{Understanding the user}

\begin{itemize}
\item
  Common mistake: describing what you WANT your users to be, rather than
  what they are
\item
  \textbf{The user is always right in behavior}. If they fail in doing
  something with your interface, the interface was not easy to use. They
  are not good at giving suggestions, but they are never wrong in their
  reactions.
\end{itemize}
\textbf{WHO} is your user

\textbf{WHEN} is it used

\textbf{WHERE} is it used

\textbf{WHAT} features are a must / what is optional

(break)

(30 minutes) Identify customer COMPREHENSIVELY and create a CUSTOMER
PROFILE. For each possible customer segment, make sure to answer the 4
questions above in as much detail as possible. But for the WHO
especially. Old/young/internet literacy/language/income/education/car
owner/pet owner/where do they go regularly/special characteristics. As
you are doing this, think vaguely about what kinds of solutions are
better for them: mobile, web, integration to a currently-existing
product (e.g.~FB game, car bumper sticker)

(rest of time, depending on how much there is left) Now that you have a
comprehensive customer profile, pretend that you are building a feature
/ app / website / whatever for them. Prototype this on paper, and
role-play small features. Each group should have 2 paper prototype
copies, and pair off half-groups with each other to test each other's
product. Each tester group should be given a comprehensive customer
profile, and understand the customer segment / problem / area. Give
feedback pretending you are that customer segment. Testing team takes
comprehensive notes on how the person uses their product / whatever.
Later, will take this feedback and integrate it into their next-version
prototype.
